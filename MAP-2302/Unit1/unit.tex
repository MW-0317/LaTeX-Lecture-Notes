\section{Lecture 1: What are differential equations?}

Remember that differential equations are equations defined by equations with derivatives. One of the simplest examples, with variables x and y, are:
\begin{align*}
    dy=dx
\end{align*}
Where the initial function, through integration can be found as:
\begin{align*}
    y=x
\end{align*}

The differential equation learned from Calc 1 is the one that describes a population, where:
\begin{align*}
    \frac{dP}{dt}=kP
\end{align*}
Where the derivative, or rate of change, is dependent on the \emph{current population}.

\section{Lecture 2}
\begin{example}{Excplicit Solution}{}
    Now let's look at an explicit solution. \par
    Verify that $\phi = 3\sin 2x + e^{-x}$ is a solution of $y'' + 4y = 5e^{-x}$.
    \begin{align*}
        \phi ' = 6\sin {2x} - e^{-x} \\
        \phi '' = -12\sin {2x} + e^{-x} 
    \end{align*}
    Now we plug in
    \begin{gather*}
        -12\sin {2x} + e^{-x} + 12\sin 2x + 4e^{-x} = 5e^{-x} \\
        5e^{-x} = 5e^{-x} \checkmark
    \end{gather*}
\end{example}

\begin{theorem}{Existence and Uniqueness}{}
    If $f(x,y)$ and $\frac{\partial f}{\partial y}(x,y)$ are continuous about the point $(x_0, y_0)$ then the Initial Value Problem $y' = f(x,y)$ and $y(x_0) = y_0$ has a unique solution in a neighborhood of the point $(x_0,y_0)$.
\end{theorem}

\begin{example}{}{}
    $y' = xy^{\frac 12} = f(x,y)$. \par
    Clearly $f(x,y)$ is continuous about $(0,0)$ but $\frac{\partial f}{\partial y} = \frac{x}{2y^{\frac 12}}$ is not continuous at $(0,0)$. So the theorem cannot say that there is a unique solution. \par
    Lets say $y_1 = 0$ for every $x$:
    \begin{align*}
        y_1' = 0 \quad & \quad xy^{\frac 12} = 0 
    \end{align*}
    \begin{gather*}
        y' = xy^{\frac 12} \checkmark
    \end{gather*}
    Lets now try $y_2 = \frac{x^4}{16}$
    \begin{align*}
        y_2' = \frac{x^3}{4} & xy^{\frac 12} = x(\frac{x^4}{16})^{\frac 12} = \frac{x^3}{4}
    \end{align*}
    \begin{gather*}
        y' = xy^{\frac 12} \checkmark
    \end{gather*}
\end{example}

\section{Lecture 3}
Let's say $y'=f(x,y)$. Try to get an idea of how the solution curves. Now, remember the interpretation of $y'$: is it the slope the tangent line. Now let's plot plenty of small tangent lines along some graph for an equation of f(x,y).

% TODO: Implement graph that does above.

This is defined as a \textbf{Direction Field}.

\begin{definition}{Direction Field}{}
    A field of vectors or slopes that represent a function at any given set of points.
\end{definition}

\begin{example}{$y'=x^2=f(x,y)$}{}
    
\end{example}

\begin{example}{$y'=\frac xy$}{}

\end{example}

\begin{definition}{Isoclines}{}
    \textbf{Isoclines} are curves of \emph{equal} slope. Isoclines do not intersect unless $f(x,y)$ is not defined at the point. Isoclines are used to develop vector fields or directional fields.
\end{definition}

We can use isoclines to create a direction field. To do so we set the derivative or $y'$ to m and solve for $y$.

\begin{example}{$y'=\frac xy=m$ ; $y=\frac 1m x$}{}
    
\end{example}

\begin{example}{$y'=-\frac xy=m$ ; $y=-\frac 1m x$}{}
    
\end{example}