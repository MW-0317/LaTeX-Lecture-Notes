\section{What are differential equations?}

Remember that differential equations are equations defined by equations with derivatives. One of the simplest examples, with variables x and y, are:
\begin{align*}
    dy=dx
\end{align*}

Where the initial function, through integration can be found as:
\begin{align*}
    y=x
\end{align*}

The differential equation learned from Calc 1 is the one that describes a population, where:
\begin{align*}
    \frac{dP}{dt}=kP
\end{align*}

Where the derivative, or rate of change, is dependent on the \emph{current population}. We can also take how a bank might offer an interest rate, consider below an interest rate of $3\%$.
\begin{align*}
    \frac{dy}{dx}=0.03y
\end{align*}
A solution to this equation is $y(t)=e^{0.03t}$ which we can prove by inputing $y(t)$ and it's derivative: $y'(t)=0.03e^{0.03t}$
\begin{align*}
    0.03e^{0.03t} = 0.03e^{0.03t} \checkmark
\end{align*}
With this solution we can also find that, for any real value of $C$, the solution still proves correct ($y(t)=Ce^{0.03t}$, $y'(t)=C(0.03)e^{0.03t}$).
\begin{align*}
    C(0.03)e^{0.03t}=Ce^{0.03t} \checkmark
\end{align*}
Which means that there are \emph{infinitly} many solutions, which we will find true for many differential equations.

Thus we must ask
\begin{enumerate}
    \item When do we have solutions?
    \item If we do, how many?
    \item And how do we find these solutions for any given differential equation?
\end{enumerate}

\subsection{Key Definitions}

We define ODE's as follows:
\begin{definition}{Orinary Differential Equations}{}
    \begin{align*}
        y^{(n)} = f(t, y, y', y'', ..., y^{(n-1)})
    \end{align*}
    The \textbf{\emph{order}} is the highest derivative of the function.
\end{definition}

Well then we ask how do we solve for these ODE's?

\begin{example}{}{}
    Lets first look at a simple example: $y''-4y'+3y=0$. We can actually solve this with $y_1(t) = e^t$ or $y_2(t) = C_1e^t$. Or even $y_3(t) = C_1e^{3t}$. A \textbf{\emph{general solution}}, a solution that contains all possible solutions, would be $y(t)=C_1e^t + C_2e^{3t}$. But then where did these two equations derive from? 
\end{example}

\begin{definition}{}{}
    A solution to an ODE is defined as:
    \begin{align*}
        \phi^{(n)} = f(t, \phi, \phi', \phi'', ..., \phi^{(n-1)})
    \end{align*}
\end{definition}

Once we find the solution to the ODE, it can then be found the \emph{exact} equation using initial conditions where the initial conditions are defined as follows:
\begin{definition}{}{}
    \begin{align*}
        y(0) = y_0, y'(0) = y_0', ..., y^{(n-1)}(0) = y_0^{(n-1)}
    \end{align*}
    This will then solve for the \textbf{\emph{Initial Value Problem}}.
\end{definition}

\section{Separation of Variables}

Lets take the \textbf{Exponential Growth} equation: $\frac{dy}{dt} = ky$. This equation can model many real world ideas like population growth, virus spread, compound interest, etc. It's a very powerful equation. The solution is as follows: $y=C_1e^{kt}$. We can find this solution by separating the variables.
\begin{align*}
    \frac{dy}{dt} & = ky \\
    \frac{1}{y} \frac{dy}{dt} & = k \\
    \int \frac{1}{y} \frac{dy}{dt} dt & = \int k dt \\
    \int \frac{1}{y} dy & = \int k dt \\
    ln|y| & = kt + C \\
    y & = e^{kt + C} = e^{C}e^{kt} = C_1e^{kt} \\
\end{align*}

Generically we solve through separation of variables by:
\begin{align*}
    \frac{dy}{dt} & = g(y)f(t) \\
    \frac{1}{g(y)} \frac{dy}{dt} & = f(t) \\ 
    \int \frac{1}{g(y)} \frac{dy}{dt} dt & = \int f(t) dt \\
    \int \frac{1}{g(y)} dy & = \int f(t) dt
\end{align*}
Sometimes these solutions can be implicit and cannot be solved for $y$ or $x$.

\subsection{Newton's Law of Cooling}
We can model cooling or heating of an object in a room with the following:
\begin{align*}
    \frac{dT}{dt} = -k (T-A)
\end{align*}
Where $T$ is the temperature of the object, $A$ is the ambient temperature or the temperature of the room, and $k$ is some constant. \par
We can solve for T through the separation of variables.
\begin{align*}
    \frac{dT}{dt} & = -k (T-A) \\
    \frac{1}{T-A} & = -k \\
    \int \frac{1}{T-A} dT & = \int -k dt \\
    ln|T-A| & = -kt + C \\
    T - A & = e^{-kt + C} \\
    T & = C_1e^{-kt} + A
\end{align*}

\section{Geometric Meaning \\ \normalsize \normalfont \emph{Slope Fields, Integral Curves, and Isoclines}}

Slope fields are graphs of small slopes plotted to help visualize a differential equation. We can either plot them directly or use \textbf{\emph{Isoclines}} to graph the DE. We can find these Isoclines by finding when the slope equals some integer (usually from -2 to 2) and plotting the curves with their integer slopes.
\textbf{\emph{Integral Curves}} are then the \emph{solutions} that follow the lines in the slope field.

\section{Existance and Uniqueness}
Given an Initial Value Problem: $y' = f(x,y), y(x_0) = y_0$
\begin{enumerate}
    \item Does a solution \emph{exist}?
    \item Is that solution \emph{unique}?
\end{enumerate}

\begin{theorem}{}{}
    If $f$ and $\frac{df}{dy}$ are continuous near $(x_0, y_0)$, then there is a unique solution on an interval $\alpha < x_0 < \beta$ to the I.V.P.
    \begin{align*}
        y' = f(x,y), y(x_0) = y_0
    \end{align*}
    If $f$ is continuous then it guarantees existance only.
\end{theorem}

\section{Linear Differential Equations}
A linear differential equation is a ordinary differential equation where $y, y'', y''',...$ are all linear with respect to $y$. Or if it follows: 
\begin{align*}
    a_n(x)y^{(n)} + a_{n-1}y^{(n-1)} + ... a_0(x)y = b(x)
\end{align*}
A Linear ODE is Homogeneous if $b(x) = 0$.
\par
Let's take a first order linear differential equation like the following $y' + p(x)y = f(x)$. How might we solve this equation? Let's first multiply a third function, $r(x)$ on both sides of the equation and we get:
\begin{align*}
    r(x)y' = r(x)p(x)y = r(x)f(x)
\end{align*}
Is it then possible to force the left hand side to be the derivative of some function of $x$ multiplied by $y$? The left hand side \emph{seems} to be comparable to the product rule. Let's see if this follows:
\begin{align*}
    \frac{d}{dx}[r(x)y] = r(x)y' + r'(x)y
\end{align*}
Now we see some similarities, the only differences actually are that of the functional coefficients of $y$. Well let's see what happens when we set them equal to each other.
\begin{align*}
    r'(x) = r(x)p(x)
\end{align*}
Now we can solve this as a separable equation.
\begin{align*}
    \frac{1}{r(x)} \frac{dr}{dx} & = p(x) \\
    \int \frac{1}{r(x)} dr & = \int p(x) dx \\
    ln|r(x)| & = \int p(x) dx \\
    r(x) & = e^{\int p(x) dx}
\end{align*}
Now that we know that an $r(x)$ exists we can use it in our differential equation
\begin{align*}
    \frac{d}{dx}[r(x)y] = r(x)f(x)
\end{align*}
Then lastly integrating $dx$ we find that 
\begin{align*}
    y & = \frac{1}{r(x)}\int r(x) f(x) dx \\
    r(x) & = e^{\int p(x) dx}
\end{align*}

\begin{theorem}{Existance and Uniqueness for Linear}{}
    If $f(x)$ and $p(x)$ are continuous on $(a,b)$ then a solution exists and is unique on $(a,b)$.
\end{theorem}

\textbf{Note that everything after here is based on my prof's teachings and not the lecture videos.}

\section{Lecture 2}
\begin{example}{Excplicit Solution}{}
    Now let's look at an explicit solution. \par
    Verify that $\phi = 3\sin 2x + e^{-x}$ is a solution of $y'' + 4y = 5e^{-x}$.
    \begin{align*}
        \phi ' = 6\sin {2x} - e^{-x} \\
        \phi '' = -12\sin {2x} + e^{-x} 
    \end{align*}
    Now we plug in
    \begin{gather*}
        -12\sin {2x} + e^{-x} + 12\sin 2x + 4e^{-x} = 5e^{-x} \\
        5e^{-x} = 5e^{-x} \checkmark
    \end{gather*}
\end{example}

\begin{theorem}{Existence and Uniqueness}{}
    If $f(x,y)$ and $\frac{\partial f}{\partial y}(x,y)$ are continuous about the point $(x_0, y_0)$ then the Initial Value Problem $y' = f(x,y)$ and $y(x_0) = y_0$ has a unique solution in a neighborhood of the point $(x_0,y_0)$.
\end{theorem}

\begin{example}{}{}
    $y' = xy^{\frac 12} = f(x,y)$. \par
    Clearly $f(x,y)$ is continuous about $(0,0)$ but $\frac{\partial f}{\partial y} = \frac{x}{2y^{\frac 12}}$ is not continuous at $(0,0)$. So the theorem cannot say that there is a unique solution. \par
    Lets say $y_1 = 0$ for every $x$:
    \begin{align*}
        y_1' = 0 \quad & \quad xy^{\frac 12} = 0 
    \end{align*}
    \begin{gather*}
        y' = xy^{\frac 12} \checkmark
    \end{gather*}
    Lets now try $y_2 = \frac{x^4}{16}$
    \begin{align*}
        y_2' = \frac{x^3}{4} & xy^{\frac 12} = x(\frac{x^4}{16})^{\frac 12} = \frac{x^3}{4}
    \end{align*}
    \begin{gather*}
        y' = xy^{\frac 12} \checkmark
    \end{gather*}
\end{example}

\section{Lecture 3}
Let's say $y'=f(x,y)$. Try to get an idea of how the solution curves. Now, remember the interpretation of $y'$: is it the slope the tangent line. Now let's plot plenty of small tangent lines along some graph for an equation of f(x,y).

% TODO: Implement graph that does above.

This is defined as a \textbf{Direction Field}.

\begin{definition}{Direction Field}{}
    A field of vectors or slopes that represent a function at any given set of points.
\end{definition}

\begin{example}{$y'=x^2=f(x,y)$}{}
    
\end{example}

\begin{example}{$y'=\frac xy$}{}

\end{example}

\begin{definition}{Isoclines}{}
    \textbf{Isoclines} are curves of \emph{equal} slope. Isoclines do not intersect unless $f(x,y)$ is not defined at the point. Isoclines are used to develop vector fields or directional fields.
\end{definition}

We can use isoclines to create a direction field. To do so we set the derivative or $y'$ to m and solve for $y$.

\begin{example}{$y'=\frac xy=m$ ; $y=\frac 1m x$}{}
    
\end{example}

\begin{example}{$y'=-\frac xy=m$ ; $y=-\frac 1m x$}{}
    
\end{example}