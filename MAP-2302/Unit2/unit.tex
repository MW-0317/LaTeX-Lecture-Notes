\section{Introduction to the uses of the first DE's}

\begin{example}{Gravity}{}
    
\end{example}

\section{Separable Equation}

\begin{definition}{Separable Equations}{}
    A differential equation is separable if $y'=f(x,y)=g(x)p(y)$
\end{definition}

\begin{example}{Is $y'=e^{x+y}$ a separable equation?}{}
    We find that $y'=f(x,y)=(e^x)(e^y)$ where $g(x)=e^x$ and $p(y)=e^y$. Therefor it is a separable equation.
\end{example}

\begin{definition}{Separable Equations and Integrals}{}
    \begin{gather}
        \frac{dy}{dx}=f(x,y)=g(x)p(y) \\
        \frac 1{p(y)} \frac{dy}{dx} = g(x)
    \end{gather}
    Let $h(y(x))=p^{-1}(y(x))$
    \begin{gather}
        h(y(x))\frac{dy}{dx}=g(x)
    \end{gather}
    Let $H(y(x)), G(x)$ be antiderivatives of $h(y(x)), g(x)$, respectively.
    \begin{gather}
        \frac{dH}{dy}\frac{dy}{dx}=\frac{dG}{dx} \\
        \frac{dH}{dx}=\frac{dG}{dx}
    \end{gather}
\end{definition}

\section{Linear Equations}

\begin{definition}{Linear Equation Definition}{}
    Let's first define linear functions as below
    \begin{gather*}
        a_1(x)\frac{dy}{dx} + a_2(x)y = b(x)
    \end{gather*}
    or
    \begin{align*}
        \frac{dy}{dx} + p(x)y = q(x) && \text{ where } p(x) = \frac{a_2(x)}{a_1(x)} \quad q(x) = \frac{b(x)}{a_1(x)}
    \end{align*}
\end{definition}

\subsection{How do we solve linear equations?}

Take from the definition above and multiply my $\mu(x)$.
\begin{gather*}
    \mu(x)\frac{dy}{dx} + \mu(x) p(x)y = \mu(x) q(x)
\end{gather*}
Assume that $\frac{d\mu}{dx}=\mu(x)p(x)$. Which we can find by doing:

\begin{align*}
    \frac{1}{\mu(x)}\frac{d\mu}{dx} &= p(x)dx \\
    \frac{d}{dx}\left[ \ln (\mu(x)) \right] &= p(x) \\
    ln(\mu(x)) &= \int{p(x) dx} \\ 
    \mu(x) &= e^{\int{p(x) dx}}
\end{align*}
However, now with $\mu(x)$ the function can be simplified to a much easier to understand form.

\begin{align*}
    \mu(x)\frac{dy}{dx} + \frac{d\mu}{dx}y &= \mu(x)q(x) \\
    \frac{d}{dx}\left[ \mu(x)y \right] &= \mu(x)q(x) \\
    y &= \frac1{\mu(x)} \int{\mu(x)q(x)}dx
\end{align*}
With both equations combined:

\begin{gather*}
    y = \frac1{e^{\int{p(x) dx}}} \int{e^{\int{p(x) dx}}q(x)}dx
\end{gather*}

\section{Exact Equations}

Let $F(x, y(x)) = 0$ be an implicit solution of a differential equation. Find $\frac{dy}{dx} = f$

\begin{gather*}
    \frac{d}{dx}\left[ F(x,y(x)) \right] = \frac{\partial F}{\partial y} + \frac{\partial F}{\partial y} \frac{dy}{dx} = 0 \\
    \frac{dy}{dx} = m -\frac{F_x}{F_y}
\end{gather*}

\begin{definition}{}{}
    $\frac{dy}{dx}$ is said to be exact if there exists an $F(x,y(x))$ such that $f=-\frac{F_x}{F_y}$
\end{definition}

\begin{example}{}{}
    $\frac{dy}{dx} = \frac{2xy}{1 + y}$, is it exact?
\end{example}

\begin{theorem}{}{}
    Let $f=-\frac MN$ or $\frac{dy}{dx} = -\frac MN$. $Ndy = -Mdx$ or $Mdx + Ndy=0$. $\frac{dy}{dx} = f$ is exact iff $\frac{\partial M}{\partial y} = \frac{\partial N}{\partial x}$
\end{theorem}